\chapterimage{container.jpg}

\chapter{Is this yours?}\label{sec:packets}

Information travels through this graph of networks in the form of packets, which contain a small amount of 
information and meta-information. 

Their format must be agreed upon by both ends of the communication, so that those packets can be automatically produced 
and interpreted.

When packets travel through one or more networks, they can get lost or reordered. If a continuous stream of data must 
flow between two computers, packets must contain mechanisms to detect losses and restore the proper order.

\section{I have a delivery for you}\label{sec:packets:lan_wan}

It is not feasible to establish physical connections between each pair of computers that want to communicate.
Instead, relatively few connections are shared to carry messages between many pairs of devices.

If a single device transmits data continuously for a long time, that connection is blocked and may become a bottleneck that 
prevents other devices to communicate. To avoid this problem, connections limit the maximum amount of data
that can be sent without interruption. As a result, devices are forced to send data in discrete bursts called 
\conceptRef{packet}{packets}. The exchange of these type of messages across one or more networks is called
\concept{packet switching}.

\begin{remark}
The concept of \concept{packet switching} (crucial to the creation of the Internet)
was developed in the 1960's (during the cold war) by the US military
to have a communications network that would work in case of a nuclear strike.
 
The problem with the one they had was that, if one station was damaged, the whole system stopped working.
However, by using packet switching, any connected subgraph of the network would keep working.

The reasoning for creating a packet-switched network was that they could still retaliate their enemy in case of an attack, which would
make them a less attractive target.

Leonard Kleinrock, one of the pioneers of internetworking, disputes the authorship of packet switching,
which is often attributed to Paul Baran and Donald Davies.
\end{remark}

\conceptRef{packet}{Packets} need to be small so that connections are not blocked for too long. 
At the same time, we want packets to contain as much data as possible, because each packets 
contain \concept{overhead} data (such as the addresses discussed in Section~\ref{sec:where}) that need
to be sent in addition to the user's payload data. Networks set the maximum packet length
using a parameter called \conceptRef{MTU}{Maximum Transmission Unit} (MTU), \eg, 1500~bytes in Ethernet.


Packet transmission across a \concept{data link} must be done carefully,
particularly when that link is a \concept{bus} shared by multiple devices.
Each packet must be individually distinguishable from the rest, 
which can be done via at least three strategies:
\begin{enumerate}
\item \textit{Fixed length}: the length of all packets is the same. The sender and the receiver must have
agreeded to this value in advance.

\item \textit{Explicit length}: the length of the packet is included in the packet, \eg, 
using the first byte of the packet. The sender and the receiver must agree on how this information
is included and how to interpret it.

\item \textit{Escape sequences}: Certain binary \conceptRef{escape sequence}{escape sequences} 
within the data have a special meaning. 
For instance, one could define the byte \otherBase{11110000} (\otherBase{0xf0}) 
to mean ``end of packet'': these bits must appear after each packet, and only then.
% 
Then the sender could start transmitting the contents of a packet, followed by \otherBase{0xf0}.
Both parties must agree on what escape sequences to use, what they mean, and how to 
send data equal to those sequences (\eg, it should be possible to send \otherBase{0xf0f0f0}
without triggering an ``end of packet'' until all bytes are transmitted).
\end{enumerate}

\begin{remark}
It is also possible to to signal the \textit{beginning} of a packet. In that case, a preceding sequence called \concept{preamble} is used.
\end{remark}


\begin{exercise}
All strategies described above are used nowadays in real scenarios, depending on the 
features, costs and trade-offs of each one.

Discuss \textbf{which of these strategies would be most appropriate} for each of these scenarios:
\begin{itemize}
\item A single manufacturer designs the hardware and software that communicates two endpoints.
A single \concept{data line} must be \conceptRef{multiplex}{multiplexed} for multiple control commands
as well as multiple data streams. The priority is efficient power and buffer usage.

\item Multiple devices share a \concept{bus}. They occasionally send messages of different lengths,
but not a huge volume of data. The priority is cost.

\item Multiple devices share a \concept{bus}. They continuously send messages of different lengths,
trying to maximize the effective transmission rate through the \concept{channel}. 
The priority is \concept{throughput}.
\end{itemize}
\end{exercise}

\section{Please fill in the form}\label{sec:packets:format}

Regardless of the strategy employed, data packets typically comprise two parts: 
\begin{itemize}
 \item Data \concept{payload}: the information that needs to be transmitted, and
 \item \conceptRef{metadata}{Metadata}: the meta-information needed to deliver the payload.
\end{itemize}
Length might be one of the metadata \conceptRef{field (packet)}{fields} included in the packet, along with 
others that help identify their type and function.
% 
The location of payload and metadata, as well as the exact metadata fields included, their length and their meaning
must be known by both ends of the communication. All these decisions constitute the \concept{packet format}
or \concept{packet structure}.

One of the most common ways of arranging payload and metadata is with a \concept{header}-\concept{body} format.
In this case, all meta-information is included first, followed by the data. 
Other formats may include a \concept{trailer} (also known as \concept{tail}, most often used for \concept{error detection}
and correction), in combination with the header, or replacing it.
All of the following configurations are possible.

\begin{minipage}{\linewidth}
\begin{center}
\begin{bytefield}{32}
\bitbox{11}{Header} & \bitbox{21}{Body}\\[-0.35cm]
\end{bytefield}
% 
\begin{bytefield}{32}
\bitbox{6}{Header} & \bitbox{19}{Body} & \bitbox{7}{Trailer/Tail}\\[-0.35cm]
\end{bytefield}
% 
\begin{bytefield}{32}
\bitbox{21}{Body} & \bitbox{11}{Trailer/Tail}\\[-0.35cm]
\end{bytefield}
% 
\begin{bytefield}{32}
\bitbox{32}{Header}
\end{bytefield}
\end{center}
\end{minipage}


The format of the payload in a packet is normally user-defined. The format of the header, however,
is strictly defined so that it can be easily produced and \conceptRef{parsing}{parsed}.
This format is often presented using a diagram that helps identify the individual bit and byte positions,
like in the following (not-real) example:\\

\begin{center}
\begin{bytefield}{32}
\bitheader{0-31} \\
\begin{rightwordgroup}{Header}
\bitbox{16}{Length} & \bitbox{8}{Importance} & \bitbox{4}{Type} & \bitbox{4}{Mode}\\
\bitbox{24}{Destination Address} & \bitbox{8}{(Reserved)} 
\end{rightwordgroup} \\
\begin{rightwordgroup}{Body}
\wordbox[lrt]{1}{Payload} \\
\skippedwords \\
\wordbox[lrb]{1}{} 
\end{rightwordgroup}
\end{bytefield}
\end{center}

\begin{remark}
Fields may also have fixed or variable lengths, which do not necessarily align with 
\conceptRef{byte boundary}{byte boundaries}. Fields may also have length equal to $1$~bit,
in which case its is normally called a \concept{flag} or flag bit.d
\end{remark}


\begin{remark}
In diagrams like the one above, multiple lines (rows) are often used so that they can be
easily printed and read. In that case, the meaning is the same as if all fields had been
presented along the same line (row).
\end{remark}

\begin{exercise}
The following packet contents were \conceptRef{sniff}{sniffed} out of a \concept{network},
expressed in \concept{hexadecimal}. Assuming the packet has the format of the example above:
% 
\begin{itemize}
\item \textbf{Indicate the value of each field} (length, importance, type, mode, destination address, and payload.
\item Can you guess the meaning of the payload?
\end{itemize}
% 
\begin{center}
\otherBase{00 0c 00 f3 bb bb bb 00 68 65 6c 70}
\end{center}
\end{exercise}

\begin{exercise}
The following code accepts two arguments: (1, \inlineCode{sys.argv[1]}) the \concept{path} of an input file, 
and (2, \inlineCode{sys.argv[2]}) the \concept{path} of an output file.
The first 255 bytes of the contents of the input file are read, 
they are formatted as a packet, and the bytes of that packet
are output to the output file.
% 
\begin{itemize}
\item Describe the \concept{packet format} used in the code, and its limitations.

\item Extend the packet format so that 
 \begin{itemize}
   \item Packets can be longer than 255 bytes
   \item The \concept{address} of the destination device can be encoded
  \end{itemize}
  
\item Provide a \concept{byte diagram} of the new format, 
as well as an explanation of the addressing system you are using.

\item Modify the code so that it outputs packets of your proposed format.
\end{itemize}
\end{exercise}

\begin{center}
\showCode{snippets/simplepacketoutput.py}
\end{center}

\begin{remark}
Languages like Python can make a distinction between files containing text and files containing binary data.
In the code above, files are open in \concept{binary mode} using modes \inlineCode{'rb'} and \inlineCode{'wb'}.

When open in binary mode, file bytes are directly represented by a \inlineCode{bytes} object, an array of 
integers between $0$ and $255$. In \concept{text mode}, those bytes are processed (\conceptRef{decode}{decoded}) further by the \inlineCode{open} method,
and returns a string that might contain special characters like accents or non-latin glyphs. \readMore{https://docs.python.org/3.12/library/functions.html\#open}
\end{remark}

\section{They keep coming}\label{sec:packets:stream}

When developing applications that use networking capabilities, it is often useful 
to imagine communication as a \concept{stream}, \ie, as a conceptual pipe where we put 
our data (any amount of data) in one end, and it comes out at the other end. If we have
this capability, then it becomes much easier to send files of any size, and to transmit a never-ending
amount of audio or video. Unfortunately, all we have to simulate those streams are packets.

\concept{Packets} often need to be forwarded through multiple networks. \conceptRef{router}{Routers} are not perfect
and are sometimes inoperative or improperly configured, so not all packets arrive to their destination. Moreover,
different packets may be delivered through different routes that may be of different length, so packets may arrive
out of order.

If we want to simulate a \concept{stream} of data, packets must contain enough meta-information in them so that losses can be detected and 
the correct order can be restored. This introduces an \concept{overhead} that is not suitable in all scenarios --\eg, very low latency--,
so some applications base their network communication in \conceptRef{message}{messages} instead of streams.

When streams are required, the concept of \concept{offset} is used to reassemble multiple messages into 
a single stream. Each packet contains some bytes of the data stream, and the offset indicates the exact location in that stream.

\begin{exercise}

The following diagram describes a stream of 48 bytes produced by a source node, 
as well as the packets it was split into before transmission:\\[-0.25cm]

\begin{center}
\begin{bytefield}[bitheight=2cm]{48}
\bitheader{0-47} \\
\bitbox{10}{\underline{Packet 0} \\Offset 0 \\Length 10}
\bitbox{14}{\underline{Packet 1} \\Offset 10\\Length 14}
\bitbox{6 }{\underline{Packet 2} \\Offset 24\\Length 6 }
\bitbox{18}{\underline{Packet 3} \\Offset 30\\Length 18}
\end{bytefield}
\end{center}

\begin{itemize}
\item Describe the minimum header that can be used to transmit this stream.
\item Assuming that the recipient receives Packet 2 first, followed by Packet 0 and other packets are lost: \textbf{what bytes of the stream are known} by the 
  destination node, and what bytes are unknown?
\item How can the source node know that some packets were not delivered?
\item What would happen if a bogus message is received by the destination, with offset $7$~bytes and length $4$~bytes?
\end{itemize}

\end{exercise}



% Two mechanisms are often used for this purpose: \conceptRef{sequence number}{sequence numbers} and \conceptRef{offset}{offsets}:
% \begin{itemize}
% \item 
% \end{itemize}




























