\usepackage[top=1.5cm,bottom=1cm,left=2cm,right=2cm,headsep=10pt,a4paper]{geometry} % Page margins

\usepackage[dvipsnames]{xcolor}
\definecolor{color1}{RGB}{243,102,25} % Define the orange color used for highlighting throughout the book
\definecolor{color2}{RGB}{0,100,140} % Define the orange color used for highlighting throughout the book
\definecolor{color3}{RGB}{40,100,35} % Define the orange color used for highlighting throughout the book
\definecolor{color4}{RGB}{243,102,25} % Define the orange color used for highlighting throughout the book

\usepackage{imakeidx} % Required to make an index (needed BEFORE hyperref)
\indexsetup{level=\section*,noclearpage}
\makeindex[columns=3,title={}] % Tells LaTeX to create the files required for indexing
\usepackage{hyperref}
\hypersetup{hidelinks,colorlinks=true,linktoc=all,breaklinks=true,allcolors=color2,bookmarksopen=false,pdftitle={Study Guide: \subjectName},pdfauthor={Miguel Hernández-Cabronero <miguel.hernandez@uab.cat>}}
\usepackage{pdfcomment}


\usepackage{graphicx} % Required for including pictures
\graphicspath{{style/}{chapters/fig/}} % Specifies the directory where pictures are stored

\usepackage[utf8]{inputenc} % Required for including letters with accents
\usepackage[LGR,T1]{fontenc}
\newcommand{\textgreek}[1]{\begingroup\fontencoding{LGR}\selectfont#1\endgroup}
\usepackage{todonotes}
\usepackage{schedule}
\usepackage{multicol}
\usepackage{pgf-pie}
\usepackage[skip=6pt plus1pt, indent=0pt]{parskip}
\usepackage{comment}
\usepackage{xifthen}
\usepackage{wrapfig}
\usepackage{circledtext}
\usepackage{textcomp}
\usepackage{moresize}
\usepackage{lipsum} % Inserts dummy text
\usepackage{tikz} % Required for drawing custom shapes
\usepackage[english]{babel} % English language/hyphenation
\usepackage{enumitem} % Customize lists
\setlist{nolistsep} % Reduce spacing between bullet points and numbered lists
\usepackage{booktabs} % Required for nicer horizontal rules in tables
\usepackage{listings}
\usepackage{minted}
\usepackage{circledsteps}
\usepackage{bytefield}
\usepackage{qrcode}


%----------------------------------------------------------------------------------------
%	FONTS
%----------------------------------------------------------------------------------------

\usepackage{avant} % Use the Avantgarde font for headings
%\usepackage{times} % Use the Times font for headings
% \usepackage{mathptmx} % Use the Adobe Times Roman as the default text font together with math symbols from the Sym­bol, Chancery and Com­puter Modern fonts
\renewcommand{\familydefault}{\sfdefault}

\usepackage{microtype} % Slightly tweak font spacing for aesthetics
\usepackage[T1]{fontenc} % Use 8-bit encoding that has 256 glyphs

%----------------------------------------------------------------------------------------
%	BIBLIOGRAPHY AND INDEX
%----------------------------------------------------------------------------------------

% \usepackage{csquotes}
% \usepackage[style=alphabetic,citestyle=numeric,sorting=nyt,sortcites=true,autopunct=true,autolang=hyphen,hyperref=true,abbreviate=false,backref=true,backend=biber,defernumbers=true]{biblatex}
% \left\addbibresource{bibliography.bib} % BibTeX bibliography file
% \defbibheading{bibempty}{}

\usepackage{calc} % For simpler calculation - used for spacing the index letter headings correctly

%----------------------------------------------------------------------------------------
%	MAIN TABLE OF CONTENTS
%----------------------------------------------------------------------------------------

\usepackage{titletoc} % Required for manipulating the table of contents

\contentsmargin{0cm} % Removes the default margin

% Part text styling
\titlecontents{part}[0cm]
{\addvspace{20pt}\centering\large\bfseries}
{}
{}
{}

% Chapter text styling
\titlecontents{chapter}[1.25cm] % Indentation
{\addvspace{12pt}\large\sffamily\bfseries} % Spacing and font options for chapters
{\color{color1!60}\contentslabel[\Large\thecontentslabel]{1.25cm}\color{color1}} % Chapter number
{\color{color1}}  
{\color{color1!60}\normalsize\;\titlerule*[.5pc]{.}\;\thecontentspage} % Page number

% Section text styling
\titlecontents{section}[1.25cm] % Indentation
{\addvspace{3pt}\sffamily\bfseries} % Spacing and font options for sections
{\contentslabel[\thecontentslabel]{1.25cm}} % Section number
{}
{\hfill\color{black}\thecontentspage} % Page number
[]

% Subsection text styling
\titlecontents{subsection}[1.25cm] % Indentation
{\addvspace{1pt}\sffamily\small} % Spacing and font options for subsections
{\contentslabel[\thecontentslabel]{1.25cm}} % Subsection number
{}
{\ \titlerule*[.5pc]{.}\;\thecontentspage} % Page number
[]

% List of figures
\titlecontents{figure}[0em]
{\addvspace{-5pt}\sffamily}
{\thecontentslabel\hspace*{1em}}
{}
{\ \titlerule*[.5pc]{.}\;\thecontentspage}
[]

% List of tables
\titlecontents{table}[0em]
{\addvspace{-5pt}\sffamily}
{\thecontentslabel\hspace*{1em}}
{}
{\ \titlerule*[.5pc]{.}\;\thecontentspage}
[]

%----------------------------------------------------------------------------------------
%	MINI TABLE OF CONTENTS IN PART HEADS
%----------------------------------------------------------------------------------------

% Chapter text styling
\titlecontents{lchapter}[0em] % Indenting
{\addvspace{15pt}\large\sffamily\bfseries} % Spacing and font options for chapters
{\color{color1}\contentslabel[\Large\thecontentslabel]{1.25cm}\color{color1}} % Chapter number
{}  
{\color{color1}\normalsize\sffamily\bfseries\;\titlerule*[.5pc]{.}\;\thecontentspage} % Page number

% Section text styling
\titlecontents{lsection}[0em] % Indenting
{\sffamily\small} % Spacing and font options for sections
{\contentslabel[\thecontentslabel]{1.25cm}} % Section number
{}
{}

% Subsection text styling
\titlecontents{lsubsection}[.5em] % Indentation
{\normalfont\footnotesize\sffamily} % Font settings
{}
{}
{}

%----------------------------------------------------------------------------------------
%	PAGE HEADERS
%----------------------------------------------------------------------------------------

\usepackage{fancyhdr} % Required for header and footer configuration
\usepackage{lastpage}

\pagestyle{fancy}
\renewcommand{\chaptermark}[1]{\markboth{\sffamily\normalsize\bfseries\chaptername\ \thechapter.\ #1}{}} % Chapter text font settings
\renewcommand{\sectionmark}[1]{\markright{\sffamily\normalsize\thesection\hspace{5pt}#1}{}} % Section text font settings
\fancyhf{} \fancyhead[LE,RO]{\sffamily\normalsize} % Font setting for the page number in the header
\fancyfoot[C]{\raisebox{0.25cm}{\thepage\ {\color{color1}/}\ \pageref*{LastPage}}}
\fancyhead[L]{Study Guide{\color{color1}:} \subjectName}
\fancyhead[R]{\thepage\ {\color{color1}/}\ \pageref*{LastPage}}

\renewcommand{\headrulewidth}{0.5pt} % Width of the rule under the header
\addtolength{\headheight}{2.5pt} % Increase the spacing around the header slightly
\renewcommand{\footrulewidth}{0pt} % Removes the rule in the footer
\fancypagestyle{plain}{\fancyhead{}\renewcommand{\headrulewidth}{0pt}} % Style for when a plain pagestyle is specified

% Removes the header from odd empty pages at the end of chapters
\makeatletter
\renewcommand{\cleardoublepage}{
% \clearpage\ifodd\c@page\else
% \hbox{}
% \vspace*{\fill}
% \thispagestyle{empty}
\newpage
% \fi
}

%----------------------------------------------------------------------------------------
%	THEOREM STYLES
%----------------------------------------------------------------------------------------

\usepackage{amsmath,amsfonts,amssymb,amsthm} % For math equations, theorems, symbols, etc

\newcommand{\intoo}[2]{\mathopen{]}#1\,;#2\mathclose{[}}
\newcommand{\ud}{\mathop{\mathrm{{}d}}\mathopen{}}
\newcommand{\intff}[2]{\mathopen{[}#1\,;#2\mathclose{]}}
\newtheorem{notation}{Notation}[chapter]

% Boxed/framed environments
\newtheoremstyle{color1numbox}% % Theorem style name
{0pt}% Space above
{0pt}% Space below
{\normalfont}% % Body font
{}% Indent amount
{\small\bf\sffamily\color{color1}}% % Theorem head font
{\;}% Punctuation after theorem head
{0.25em}% Space after theorem head
{\small\sffamily\color{color1}\thmname{#1}\nobreakspace\thmnumber{\@ifnotempty{#1}{}\@upn{#2}}% Theorem text (e.g. Theorem 2.1)
\thmnote{\nobreakspace\the\thm@notefont\sffamily\bfseries\color{black}---\nobreakspace#3.}} % Optional theorem note
\renewcommand{\qedsymbol}{$\blacksquare$}% Optional qed square

\newtheoremstyle{blacknumex}% Theorem style name
{5pt}% Space above
{5pt}% Space below
{\normalfont}% Body font
{} % Indent amount
{\small\bf\sffamily}% Theorem head font
{\;}% Punctuation after theorem head
{0.25em}% Space after theorem head
{\small\sffamily{\tiny\ensuremath{\blacksquare}}\nobreakspace\thmname{#1}\nobreakspace\thmnumber{\@ifnotempty{#1}{}\@upn{#2}}% Theorem text (e.g. Theorem 2.1)
\thmnote{\nobreakspace\the\thm@notefont\sffamily\bfseries---\nobreakspace#3.}}% Optional theorem note

\newtheoremstyle{blacknumbox} % Theorem style name
{0pt}% Space above
{0pt}% Space below
{\normalfont}% Body font
{}% Indent amount
{\small\bf\sffamily}% Theorem head font
{\;}% Punctuation after theorem head
{0.25em}% Space after theorem head
{\small\sffamily\thmname{#1}\nobreakspace\thmnumber{\@ifnotempty{#1}{}\@upn{#2}}% Theorem text (e.g. Theorem 2.1)
\thmnote{\nobreakspace\the\thm@notefont\sffamily\bfseries---\nobreakspace#3.}}% Optional theorem note

% Non-boxed/non-framed environments
\newtheoremstyle{color1num}% % Theorem style name
{5pt}% Space above
{5pt}% Space below
{\normalfont}% % Body font
{}% Indent amount
{\small\bf\sffamily\color{color1}}% % Theorem head font
{\;}% Punctuation after theorem head
{0.25em}% Space after theorem head
{\small\sffamily\color{color1}\thmname{#1}\nobreakspace\thmnumber{\@ifnotempty{#1}{}\@upn{#2}}% Theorem text (e.g. Theorem 2.1)
\thmnote{\nobreakspace\the\thm@notefont\sffamily\bfseries\color{black}---\nobreakspace#3.}} % Optional theorem note
\renewcommand{\qedsymbol}{$\blacksquare$}% Optional qed square
\makeatother

% Defines the theorem text style for each type of theorem to one of the three styles above
\newcounter{dummy} 
\numberwithin{dummy}{section}
\theoremstyle{color1numbox}
\newtheorem{theoremeT}[dummy]{Theorem}
\newtheorem{problem}{Problem}[chapter]
\newtheorem{exerciseT}{Exercise}[chapter]
\theoremstyle{blacknumex}
\newtheorem{exampleT}{Example}[chapter]
\theoremstyle{blacknumbox}
\newtheorem{vocabulary}{Vocabulary}[chapter]
\newtheorem{definitionT}{Definition}[section]
\newtheorem{corollaryT}[dummy]{Corollary}
\theoremstyle{color1num}
\newtheorem{proposition}[dummy]{Proposition}

%----------------------------------------------------------------------------------------
%	DEFINITION OF COLORED BOXES
%----------------------------------------------------------------------------------------

\RequirePackage[framemethod=default]{mdframed} % Required for creating the theorem, definition, exercise and corollary boxes

% Theorem box
\newmdenv[skipabove=7pt,
skipbelow=7pt,
backgroundcolor=black!5,
linecolor=color1,
innerleftmargin=5pt,
innerrightmargin=5pt,
innertopmargin=5pt,
leftmargin=0cm,
rightmargin=0cm,
innerbottommargin=5pt]{tBox}

% Exercise box	  
\newmdenv[skipabove=0.25cm,
skipbelow=0,
rightline=false,
leftline=true,
topline=false,
bottomline=false,
% backgroundcolor=color1!10,
linecolor=color1!30,
innerleftmargin=5pt,
innerrightmargin=5pt,
innertopmargin=0.4cm,
innerbottommargin=0.2cm,
leftmargin=0cm,
rightmargin=0cm,
linewidth=3pt]{eBox}	

% Remark (note)
\newmdenv[skipabove=0.25cm,
skipbelow=0,
rightline=false,
leftline=true,
topline=false,
bottomline=false,
% backgroundcolor=color1!10,
linecolor=color2!30,
innerleftmargin=5pt,
innerrightmargin=5pt,
innertopmargin=0.2cm,
innerbottommargin=0.2cm,
leftmargin=0cm,
rightmargin=0cm,
linewidth=3pt]{rBox}

% Code listing
\newmdenv[skipabove=0.25cm,
skipbelow=0,
rightline=false,
leftline=true,
topline=false,
bottomline=false,
% backgroundcolor=color1!10,
linecolor=color3!30,
innerleftmargin=5pt,
innerrightmargin=5pt,
innertopmargin=0.2cm,
innerbottommargin=0.2cm,
leftmargin=0cm,
rightmargin=0cm,
linewidth=3pt]{codeBox}	

% Definition box
\newmdenv[skipabove=7pt,
skipbelow=7pt,
rightline=false,
leftline=true,
topline=false,
bottomline=false,
linecolor=color1,
innerleftmargin=5pt,
innerrightmargin=5pt,
innertopmargin=0pt,
leftmargin=0cm,
rightmargin=0cm,
linewidth=4pt,
innerbottommargin=0pt]{dBox}	

% Corollary box
\newmdenv[skipabove=7pt,
skipbelow=7pt,
rightline=false,
leftline=true,
topline=false,
bottomline=false,
linecolor=gray,
backgroundcolor=black!5,
innerleftmargin=5pt,
innerrightmargin=5pt,
innertopmargin=5pt,
leftmargin=0cm,
rightmargin=0cm,
linewidth=4pt,
innerbottommargin=5pt]{cBox}

% Creates an environment for each type of theorem and assigns it a theorem text style from the "Theorem Styles" section above and a colored box from above
\newenvironment{theorem}{\begin{tBox}\begin{theoremeT}}{\end{theoremeT}\end{tBox}}
\newenvironment{exercise}{\begin{eBox}\begin{exerciseT}}{\end{exerciseT}\end{eBox}}				  
\newenvironment{definition}{\begin{dBox}\begin{definitionT}}{\end{definitionT}\end{dBox}}	
\newenvironment{example}{\begin{exampleT}}{\hfill{\tiny\ensuremath{\blacksquare}}\end{exampleT}}		
% \newenvironment{example}{\begin{exampleT}}{\vspace{-0.25cm}\end{exampleT}}		
\newenvironment{corollary}{\begin{cBox}\begin{corollaryT}}{\end{corollaryT}\end{cBox}}	

%----------------------------------------------------------------------------------------
%	REMARK ENVIRONMENT
%----------------------------------------------------------------------------------------

\newenvironment{remark}{\par
\begin{rBox}
\begin{list}{}{
\leftmargin=0.75cm % Indentation on the left
\rightmargin=5pt}\item\ignorespaces % Indentation on the right
\makebox[-2pt]{\begin{tikzpicture}[overlay]
\node[draw=color2!60,line width=1pt,circle,fill=color2!0,font=\sffamily\bfseries,inner sep=2pt,outer sep=0pt] at (-12pt,3pt){\textcolor{Turquoise}{\textmusicalnote}};\end{tikzpicture}}
\advance\baselineskip -1pt}{\end{list}\end{rBox}} % Tighter line spacing and white space after remark

% --- Code
\newenvironment{code}[1]{\par
\begin{codeBox}
\makebox[-2pt]{\begin{tikzpicture}[overlay]
\node[draw=color3!60,line width=1pt,circle,fill=color3!0,font=\sffamily\bfseries,inner sep=2pt,outer sep=0pt,minimum size = 0.6cm] at (0.25cm,-0.1cm){\textcolor{color3}{\tiny\href{code/#1}{\underline{.py}}}};
\node[draw=color3!60,line width=1pt,circle,fill=color3!0,font=\sffamily\bfseries,inner sep=2pt,outer sep=0pt,minimum size = 0.6cm] at (1cm,-0.1cm){\textcolor{color3}{\tiny\href{code/#1.out}{\underline{.out}}}};
\node[font=\sffamily,inner sep=2pt,outer sep=0pt,minimum size = 0.6cm,anchor=west] at (1.5cm,-0.1cm){\textcolor{color3}{\small\href{code/#1}{#1}}};
\end{tikzpicture}}\\[-0.1cm]
}{\end{codeBox}}

\newenvironment{codeNoOutput}[1]{\par
\begin{codeBox}
\makebox[-2pt]{\begin{tikzpicture}[overlay]
\node[draw=color3!60,line width=1pt,circle,fill=color3!0,font=\sffamily\bfseries,inner sep=2pt,outer sep=0pt,minimum size = 0.6cm] at (0.25cm,-0.1cm){\textcolor{color3}{\tiny\href{code/#1}{\underline{.py}}}};
\node[font=\sffamily,inner sep=2pt,outer sep=0pt,minimum size = 0.6cm,anchor=west] at (0.75cm,-0.1cm){\textcolor{color3}{\small\href{code/#1}{#1}}};
\end{tikzpicture}}\\[-0.1cm]
}{\end{codeBox}}

%----------------------------------------------------------------------------------------
%	SECTION NUMBERING IN THE MARGIN
%----------------------------------------------------------------------------------------

\makeatletter
\renewcommand{\@seccntformat}[1]{\llap{\textcolor{color1}{\csname the#1\endcsname}\hspace{1em}}}                    
\renewcommand{\section}{\@startsection{section}{1}{\z@}
{-4ex \@plus -1ex \@minus -.4ex}
{1ex \@plus.2ex }
{\normalfont\large\sffamily\bfseries}}
\renewcommand{\subsection}{\@startsection {subsection}{2}{\z@}
{-3ex \@plus -0.1ex \@minus -.4ex}
{0.5ex \@plus.2ex }
{\normalfont\sffamily\bfseries}}
\renewcommand{\subsubsection}{\@startsection {subsubsection}{3}{\z@}
{-2ex \@plus -0.1ex \@minus -.2ex}
{.2ex \@plus.2ex }
{\normalfont\small\sffamily\bfseries}}                        
\renewcommand\paragraph{\@startsection{paragraph}{4}{\z@}
{-2ex \@plus-.2ex \@minus .2ex}
{.1ex}
{\normalfont\small\sffamily\bfseries}}

%----------------------------------------------------------------------------------------
%	PART HEADINGS
%----------------------------------------------------------------------------------------

% numbered part in the table of contents
\newcommand{\@mypartnumtocformat}[2]{%
\setlength\fboxsep{0pt}%
\noindent\colorbox{color1!20}{\strut\parbox[c][.7cm]{\ecart}{\color{color1!70}\Large\sffamily\bfseries\centering#1}}\hskip\esp\colorbox{color1!40}{\strut\parbox[c][.7cm]{\linewidth-\ecart-\esp}{\Large\sffamily\centering#2}}}%
%%%%%%%%%%%%%%%%%%%%%%%%%%%%%%%%%%
% unnumbered part in the table of contents
\newcommand{\@myparttocformat}[1]{%
\setlength\fboxsep{0pt}%
\noindent\colorbox{color1!40}{\strut\parbox[c][.7cm]{\linewidth}{\Large\sffamily\centering#1}}}%
%%%%%%%%%%%%%%%%%%%%%%%%%%%%%%%%%%
\newlength\esp
\setlength\esp{4pt}
\newlength\ecart
\setlength\ecart{1.2cm-\esp}
\newcommand{\thepartimage}{}%
\newcommand{\partimage}[1]{\renewcommand{\thepartimage}{#1}}%
\def\@part[#1]#2{%
\ifnum \c@secnumdepth >-2\relax%
\refstepcounter{part}%
\addcontentsline{toc}{part}{\texorpdfstring{\protect\@mypartnumtocformat{\thepart}{#1}}{\partname~\thepart\ ---\ #1}}
\else%
\addcontentsline{toc}{part}{\texorpdfstring{\protect\@myparttocformat{#1}}{#1}}%
\fi%
\startcontents%
\markboth{}{}%
{\thispagestyle{empty}%
\begin{tikzpicture}[remember picture,overlay]%
\node at (current page.north west){\begin{tikzpicture}[remember picture,overlay]%	
\fill[color1!20](0cm,0cm) rectangle (\paperwidth,-\paperheight);
\node[anchor=north] at (4cm,-3.25cm){\color{color1!40}\fontsize{220}{100}\sffamily\bfseries\@Roman\c@part}; 
\node[anchor=south east] at (\paperwidth-1cm,-\paperheight+1cm){\parbox[t][][t]{8.5cm}{
\printcontents{l}{0}{\setcounter{tocdepth}{1}}%
}};
\node[anchor=north east] at (\paperwidth-1.5cm,-3.25cm){\parbox[t][][t]{15cm}{\strut\raggedleft\color{color2}\fontsize{30}{30}\sffamily\bfseries#2}};
\end{tikzpicture}};
\end{tikzpicture}}%
\@endpart}
\def\@spart#1{%
\startcontents%
\phantomsection
{\thispagestyle{empty}%
\begin{tikzpicture}[remember picture,overlay]%
\node at (current page.north west){\begin{tikzpicture}[remember picture,overlay]%	
\fill[color1!20](0cm,0cm) rectangle (\paperwidth,-\paperheight);
\node[anchor=north east] at (\paperwidth-1.5cm,-3.25cm){\parbox[t][][t]{15cm}{\strut\raggedleft\color{color2}\fontsize{30}{30}\sffamily\bfseries#1}};
\end{tikzpicture}};
\end{tikzpicture}}
\addcontentsline{toc}{part}{\texorpdfstring{%
\setlength\fboxsep{0pt}%
\noindent\protect\colorbox{color1!40}{\strut\protect\parbox[c][.7cm]{\linewidth}{\Large\sffamily\protect\centering #1\quad\mbox{}}}}{#1}}%
\@endpart}
\def\@endpart{
% \vfil\newpage
% \if@twoside
% \if@openright
% \null
% \thispagestyle{empty}%
% \newpage
% \fi
% \fi
% \if@tempswa
% \twocolumn
% \fi
}

%----------------------------------------------------------------------------------------
%	CHAPTER HEADINGS
%----------------------------------------------------------------------------------------

\newcommand {\chapterHeaderHeight} {3cm}
\newcommand {\chapterHeaderSpace} {3.5cm}

% A switch to conditionally include a picture, implemented by  Christian Hupfer
\newif\ifusechapterimage
\usechapterimagetrue
\newcommand{\thechapterimage}{}%
\newcommand{\chapterimage}[1]{\ifusechapterimage\renewcommand{\thechapterimage}{#1}\fi}%
\def\@makechapterhead#1{%
{\parindent \z@ \raggedright \normalfont
\ifnum \c@secnumdepth >\m@ne
\if@mainmatter
\begin{tikzpicture}[remember picture,overlay]
\node at (current page.north west)
{\begin{tikzpicture}[remember picture,overlay]
\node[anchor=north west,inner sep=0pt] at (0,0) {\ifusechapterimage\includegraphics[width=\paperwidth]{\thechapterimage}\fi};
\draw[anchor=west] (\Gm@lmargin,-\chapterHeaderHeight) node [line width=2pt,rounded corners=15pt,draw=color1,fill=white,fill opacity=0.9,inner sep=15pt]{\strut\makebox[22cm]{}};
\draw[anchor=west] (\Gm@lmargin+.3cm,-\chapterHeaderHeight) node {\huge\sffamily\bfseries\color{black}\thechapter. #1\strut};
\end{tikzpicture}};
\end{tikzpicture}
\else
\begin{tikzpicture}[remember picture,overlay]
\node at (current page.north west)
{\begin{tikzpicture}[remember picture,overlay]
\node[anchor=north west,inner sep=0pt] at (0,0) {\ifusechapterimage\includegraphics[width=\paperwidth]{\thechapterimage}\fi};
\draw[anchor=west] (\Gm@lmargin,-\chapterHeaderHeight) node [line width=2pt,rounded corners=15pt,draw=color1,fill=white,fill opacity=0.9,inner sep=15pt]{\strut\makebox[22cm]{}};
\draw[anchor=west] (\Gm@lmargin+.\chapterHeaderHeight,-\chapterHeaderHeight) node {\huge\sffamily\bfseries\color{black}#1\strut};
\end{tikzpicture}};
\end{tikzpicture}
\fi\fi
\par
\vspace*{\chapterHeaderSpace}
}}

%-------------------------------------------

\def\@makeschapterhead#1{%
\begin{tikzpicture}[remember picture,overlay]
\node at (current page.north west)
{\begin{tikzpicture}[remember picture,overlay]
\node[anchor=north west,inner sep=0pt] at (0,0) {\ifusechapterimage\includegraphics[width=\paperwidth]{\thechapterimage}\fi};
\draw[anchor=west] (\Gm@lmargin,-\chapterHeaderHeight) node [line width=2pt,rounded corners=15pt,draw=color1,fill=white,fill opacity=0.9,inner sep=15pt]{\strut\makebox[22cm]{}};
\draw[anchor=west] (\Gm@lmargin+.\chapterHeaderHeight,-\chapterHeaderHeight) node {\huge\sffamily\bfseries\color{black}#1\strut};
\end{tikzpicture}};
\end{tikzpicture}
\par
\vspace*{\chapterHeaderSpace}
}
\makeatother

%----------------------------------------------------------------------------------------
%	HYPERLINKS IN THE DOCUMENTS
%----------------------------------------------------------------------------------------

\usepackage{hyperref}
% \hypersetup{hidelinks,colorlinks=false,breaklinks=true,urlcolor= color1,bookmarksopen=false,pdftitle={Title},pdfauthor={Author}}
\usepackage{bookmark}
\bookmarksetup{
open,
numbered,
addtohook={%
\ifnum\bookmarkget{level}=0 % chapter
\bookmarksetup{bold}%
\fi
\ifnum\bookmarkget{level}=-1 % part
\bookmarksetup{color=color1,bold}%
\fi
}
}
