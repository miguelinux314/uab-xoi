\newpage
\section{Mission 7: To TCP or not to TCP}\label{sec:mission7}

\begin{minipage}{0.23\linewidth}
% \vspace{-0.5cm}
\includegraphics[width=\linewidth]{shakespeare}
\end{minipage}
\hspace{0.03\linewidth}
\begin{minipage}{0.80\linewidth}
Years of consistent success have consolidated the use of IPv4 for internetworking both within Mars and with other planets.
During this time, the engineering teams of the colony have discovered the limitations of bare IPv4
and the need to use at least one of the main transport protocols when communicating with Earth's Internet.\\

\vspace{-0.25cm}
Implementing this fourth layer and adding it to the existing network stacks of each node in the colony is a daunting task. Thus, in a first phase only one of the two main transport protocols will be implemented.
%
The colony is asking for your help to quantitatively compare these two protocols and help decide which one to pick.
\end{minipage}

% \vspace{0.3cm}
\noindent
\textbf{\raisebox{0.03cm}{\warning}Warning! This mission is \raisebox{0.25em}{highly} critical}. Take your time, request as much help as needed, and keep your spirits up. Previous missions can be regarded as training for this one --- it is strongly recommended you carefully finish them before attempting this one.
\vspace{-0.1cm}

\subsection*{Report details}

The assembly has requested that you carefully consider the following points in your report:

\begin{enumerate}[itemsep=0.25cm]
\item To make your report standalone, include the colony's network topology using the CIDR format when applicable,
and specify reasonable MTU values (you may reuse the ones you used in previous missions). Complete this description by
specifying the bandwidth (link speed, in bytes per second -- let us assume it is constant).

\item One of the two candidate transport protocols uses a parameter called MSS. Indicate which one, and specify the largest
value that would allow communication between any two machines of the colony without requiring fragmentation. Use that value in the remainder of your report.

\item Pick a node $N_1$ from a base directly connected to the North Bus. Pick another node $N_2$ from a base directly connected to the South Bus. For each of the two main transport protocols, calculate the total time elapsed from when $N_1$ starts transmission to when $N_2$ receives the first byte of payload data. To do so, assume:
  \begin{itemize}
  \item If the protocol is connection-oriented, data are only transferred once the connection is established.
  \item All nodes and routers have infinite computing power, i.e., no delay is introduced by encapsulating/decapsulating data.
  \item Transmission speeds are still finite, and as you defined earlier in your report.
  \item Routing tables are optimal.
  \item All caches begin empty for each protocol, and entries remain valid once they are set.
  \end{itemize}
You are encouraged to include swimlane diagrams to support your answer.

\item Which of the transport protocols include mechanisms for congestion control? What conditions would be required to allow continuous transmission from $N_1$ to $N_2$ without waits due to congestion control?

\item Indicate three use cases, \textit{i.e.}, user applications: the first one suitable only for one of the transport protocols, the second one suitable for the other transport protocol, and the third one suitable for both.
\end{enumerate}

\vspace{-0.25cm}

\subsection*{Feedback details}

Use a similar feedback process as in the previous missions, making sure to touch on the following key aspects:
\begin{itemize}
\item Is the chosen MSS consistent with the specified MTU values?
\item Is the transmission time of each packet correctly calculated?
\item Do you see any problem with routing, fragmentation or ARP?
\item Are all elements of congestion control properly considered?
\item Are the proposed use cases well chosen?
\end{itemize}
